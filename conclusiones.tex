La aplicación de la metodología descrita deja entrever, como trasfondo, una alta variabilidad entre las configuraciones en la estructura de carencias, lo que podría llevarnos apresuradamente a la conclusión de que no existe una interacción significante entre los distintos factores determinantes de la pobreza multidimensional. Sin embargo, de forma previa a dicha inferencia, se propone realizar el estudio nuevamente, esta vez agrupando los datos en categorías que pudiesen implicar similitudes en la disposición de las incidencias de las carencias. En consecuencia, la siguiente etapa del proyecto consistirá, en primera instancia, en la aplicación de la metodología sobre los datos agrupados por tipo de hogar, criterio que define 6 categorías: unipersonal, nuclear monoparental, nuclear biparental, extenso monoparental, extenso biparental y hogar censal. Si este criterio de clasificación fuese determinante en la estructura de carencias, se podría esperar la obtención de indicadores de correlación de mayor significancia, al disminuir la variabilidad de los datos muestrales. 

Otra causante de los valores obtenidos en el estudio podría guardar relación con la naturaleza propia de los indicadores definidos por el modelo para cada carencia, pudiendo éstos no representar adecuadamente los aspectos de la pobreza para los cuales fueron concebidos. Como ejemplo particular, se pueden mencionar los criterios empleados para la obtención de los indicadores de las carencias agrupadas en la dimensión ``Redes y cohesión social ", que responden a una medida de percepción del entorno subjetiva. Además, la concepción de indicadores dicotómicos para cada carencia implica la ausencia de la representación de intensidad; esta pérdida de información podría reducir, en comparación con un modelo que establezca un intervalo continuo de valores, los indicadores obtenidos en el cálculo de las correlaciones. 


%%%%%%%%%%%%%%%%%%%% conclusiones red carencias
Observando las medidas de centralidad obtenidas de la red de carencias, se evidencia que las carencias con mayor peso estructural corresponden a \textit{seguridad} (0,452) y \textit{trato igualitario} (0,375), las que a su vez son las carencias con mayor grado de intermediación con 25,83 y 21,53 respectivamente. Es importante recordar que el peso estructural no tiene relación con las causas de la pobreza, sino que expresa, más bien, la importancia que tiene dicha carencia dentro de la red y que permite entender la dinámica interna de la pobreza. En este sentido, se podría decir que las carencias \textit{seguridad} y \textit{trato igualitario} serían buenos predictores simples de la pobreza multidimensional, y a su vez son aquellas que tienen en promedio una relación más significativa con el resto de las carencias. 
%%%%%%%%%%%%%%%%%%%%%%%%%%%%%%%%%%%%%%%%%%%%%%%%%%%

